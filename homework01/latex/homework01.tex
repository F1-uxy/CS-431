\documentclass[conference]{IEEEtran}
\usepackage{graphicx}
\usepackage{amsmath}
\usepackage{caption}
\usepackage{float}

% Title and author
\title{Performance Analysis of Pi Calculation}
\author{Charlie Flux}

\begin{document}
\maketitle

\begin{abstract}
This report presents an analysis of the performance of two methods for estimating Pi --- the Monte Carlo method and the area of a circle method --- across varying core counts on a high-performance computing system. The study evaluates the execution time, parallel efficiency, and scalability. Figures and charts illustrate the observed trends.
\end{abstract}

\section{Introduction}
The purpose of this experiment was to examine the impact of parallelization on Pi estimation methods using OpenMP.
Both time and accuracy was measured to investigate the effect of changing the number of threads, steps and slices in a system.
\section{Methodology}
Two methods for estimating Pi were used:
\begin{itemize}
    \item \textbf{Monte Carlo Method:} A random point generation method that calculates the ratio of points inside a circle.
    \item \textbf{Area of Circle Method:} Uses the integral approach to approximate Pi.
\end{itemize}
All tests ran measuring time used core counts of 1, 2, 4, 6, 8, 12, 16, 24, 32, 48, 64, 96 and 128 and a fixed 100,000,000 units of accuracy.
All tests ran measuring accuracy a fixed core count of 128 was used with units ranging from 1 - 1,000,000,000,000 measureing pi to 11 significant figures.

\subsection{Experimental Setup}
The tests were performed on a system supporting up to 128 cores. Each experiment was repeated 20 times, and the average time was recorded for each core configuration. The results are displayed as box plots to identify the distribution and variance in execution time.

\section{Results and Discussion}
\subsection{Critical vs Atomic}
Figure \ref{fig:critical_atomic} compares the execution time of critical against atomic sections of estimating Pi using the area of the circle algorithm.
This figure shows that there is an increase in performance when using 

\begin{figure}[H]
    \centering
    \caption{Critical vs Atomic Area of a Circle Estimation}
    \label{fig:critical_atomic}
\end{figure}

\subsection{Speedup and Efficiency}
The speedup achieved by each method is shown in Figure \ref{fig:speedup}. The Monte Carlo method scales efficiently with cores, while the area method plateaus as the core count increases.

\begin{figure}[H]
    \centering
    \caption{Speedup for Different Core Counts}
    \label{fig:speedup}
\end{figure}

\section{Conclusion}
The experiment demonstrates that the Monte Carlo method benefits more from parallelization, achieving near-linear speedup up to 128 cores. The area of a circle method, while also benefiting from parallelization, exhibits diminishing returns after 64 cores due to the nature of the integral computation.

\section{Future Work}
Future studies could investigate the impact of different scheduling strategies and thread affinity on performance.

\begin{thebibliography}{1}
\bibitem{omp} OpenMP Specifications, \textit{https://www.openmp.org/specifications/}, accessed on October 2024.
\end{thebibliography}

\end{document}
